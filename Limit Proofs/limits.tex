\documentclass{article}
\usepackage[utf8]{inputenc}

\title{Limits}
\author{Allen Williams }
\date{November 8 2017}

\usepackage{amsthm}
\usepackage{amsmath}
\usepackage{scrextend}
\newtheorem*{Theorem}{Theorem}
\newtheorem*{Axiom}{Axiom}
\newtheorem*{Problem}{Problem}
\renewcommand\qedsymbol{QED}

\binoppenalty=\maxdimen
\relpenalty=\maxdimen

\begin{document}

\maketitle
 
 \begin{Problem}
  47 a) $\lim_{n\to\infty}\frac{n-1}{n+1}=1$
  \begin{proof}
      Fix $\epsilon>0$ and let $N=\frac{2}{\epsilon}$.  Then $n>N$ means $n>\frac{2}{\epsilon}$, which means $\epsilon>\frac{2}{n}>\frac{2}{n+1}=\lvert\frac{n-1-(n+1)}{n+1}\rvert$ which implies $\epsilon>\lvert\frac{n-1}{n+1}-1\rvert$ when $n>N$, and $\lim_{n\to\infty}\frac{n-1}{n+1}=1$.
  \end{proof}
 \end{Problem}
 
  \begin{Problem}
  47 b) $\lim_{n\to\infty}\frac{1}{2n^2+1}=0$
  \begin{proof}
      Fix $\epsilon>0$ and let $N=\frac{1}{\sqrt{\epsilon}}$.  Then $n>N$ means $n>\frac{1}{\sqrt{\epsilon}}$, which means $\epsilon>\frac{1}{n^2}>\frac{1}{2n^2+1}=\lvert\frac{1}{2n^2+1}\rvert$. So $n>N$ implies $\epsilon>\lvert\frac{1}{2n^2+1}$ so $\lim_{n\to\infty}\frac{1}{2n^2+1}=0$.
  \end{proof}
 \end{Problem}
 
  \begin{Problem}
  47 c) $\lim_{n\to\infty}\frac{4n^3+2n}{2n^3+1}=2$
  \begin{proof}
      Fix $\epsilon>0$ and let $N=\frac{1}{\sqrt{\epsilon}}$.  Then $n>N$ means $n>\frac{1}{\sqrt{\epsilon}}$, which means $\epsilon>\frac{1}{n^2}$, or equivalently $\epsilon>\frac{2n}{2n^3}>\lvert\frac{2n-2}{2n^3+1}\rvert$. Then $\epsilon>\lvert\frac{4n^3+2n-2(2n^3+1)}{2n^3+1}\rvert$ or $\epsilon>\lvert\frac{4n^3+2n}{2n^3+1}-2\rvert$. So $n>N$ implies $\epsilon>\lvert\frac{4n^3+2n}{2n^3+1}-2\rvert$ proving that $\lim_{n\to\infty}\frac{4n^3+2n}{2n^3+1}=2$.
  \end{proof}
 \end{Problem}
 
   \begin{Problem}
  47 d) $\lim_{n\to\infty}\frac{(-1)^n}{n}=0$
  \begin{proof}
      Fix $\epsilon>0$ and let $N=\frac{1}{\epsilon}$, then $n>N$ means $n>\frac{1}{\epsilon}$ or $\epsilon>\frac{1}{n}$, or equivalently, $\lvert\frac{(-1)^n}{n}\rvert<\epsilon$. So $n>N$ implies $\lvert\frac{(-1)^n}{n}\rvert<\epsilon$, meaning $\lim_{n\to\infty}\frac{(-1)^n}{n}=0$.
  \end{proof}
 \end{Problem}
 
    \begin{Problem}
  47 e) $\lim_{n\to\infty}\frac{\sin(n)}{n}=0$
  \begin{proof}
      Fix $\epsilon>0$ and let $N=\frac{1}{\epsilon}$, then $n>N$ means $n>\frac{1}{\epsilon}$ or $\epsilon>\frac{1}{n}$, which also means $\epsilon>\lvert\frac{1}{n}\rvert$, since $\lvert\frac{1}{n}\rvert=\frac{1}{n}$.  $\lvert\frac{1}{n}\rvert\geq\lvert\frac{\sin(n)}{n}\rvert$ for all $n$ so $\epsilon>\lvert\frac{\sin(n)}{n}\rvert$ by transitivity.  Then $n>\frac{1}{\epsilon}$ implies $\lvert\frac{\sin(n)}{n}\rvert<\epsilon$ so $\lim_{n\to\infty}\frac{\sin(n)}{n}=0$.
  \end{proof}
 \end{Problem}
 \newpage
   \begin{Problem}
  47 f) $\lim_{n\to\infty}\sqrt{4n^2+n}-2n}=\frac{1}{4}$
  \begin{proof}
      Fix $\epsilon>0$ and let $N=\frac{4\epsilon+1}{2}$, then $n>N$ means $n>\frac{4\epsilon}{2}$, or $\epsilon>\frac{2n-1}{4}$, which is always positive so $\epsilon>\lvert\frac{2n-1}{4}\rvert$, or equivalently, $\epsilon>\lvert\frac{n^2}{2n}-\frac{1}{4}\rvert>\lvert\frac{n}{\sqrt{4n^2+n}+2n}-\frac{1}{4}\rvert$ which is equivalent to $\lvert\sqrt{4n^2+2n}-2n-\frac{1}{4}\rvert$.  So $n>N$ implies $\lvert\sqrt{4n^2+n}-2n-\frac{1}{4}\rvert<\epsilon$ proving $\lim_{n\to\infty}\sqrt{4n^2+n}-2n}=\frac{1}{4}$
  \end{proof}
 \end{Problem}
 
    \begin{Problem}
  48 a) $s_n=\sin(\frac{\pi n}{4})$ diverges.
  \begin{proof}
      Assume for contradiction that$s_n$ converges to a limit $s\in\mathbb{R}$ and let $\epsilon=1$.  Then by lemma there exists an $N\in\mathbb{R}$ such that for $n>N$, $\lvert\sin(\frac{\pi n}{4})-s\rvert<1$.  Consider cases $n=2+8k$ for $k\in\mathbb{Z}$ and $n=6+8k$ for $k\in\mathbb{Z}$.  Observe $\lvert1-s\rvert<1$ and $\lvert-1-s\rvert<1$ and note that $\lvert-(s+1)\rvert=\lvert s+1 \rvert$.  Then $\lvert 1-s\rvert + \lvert -(s+1)\rvert<2$ and by the triangle inequality $\lvert 1-s+s+1\rvert<2$ or $\lvert2\rvert<2$, but $2=2$, so a contradiction has been derived and the assumption that $s_n$ converges must be wrong, thus $s_n$ diverges.
  \end{proof}
 \end{Problem}
 
     \begin{Problem}
  48 b) $s_n=(-1)^n\cdot n$ diverges.
  \begin{proof}
      Assume for contradiction that$s_n$ converges to a limit $s\in\mathbb{R}$ and let $\epsilon=1$.  Then by lemma there exists an $N\in\mathbb{R}$ such that for $n>N$, $\lvert(-1)^n\cdot n -s\rvert<1$.  Consider cases $n=2k$ for $k\in\mathbb{Z}$ and $n=2k+1$ for $k\in\mathbb{Z}$, both of which can be found above any large $N$.  $\lvert n-s\rvert<1$ and $\lvert -n-s\rvert<1$, then by adding the two inequalities and invoking the triangle inequality, $\lvert 2n\rvert<\lvert n-s\rvert + \lvert -n-s\rvert<1$.  This cannot happen for any $n\in\mathbb{N}$ so the assumption that $s_n$ converges is false.
  \end{proof}
 \end{Problem}
 
      \begin{Problem}
  49 Let $t_n$ be a bounded sequence, i.e. there exists an $M$ such that $\lvert t_n\rvert\leq M$ for all $n$, and let $s_n$ be a sequnce such that $\lim_{n\to\infty}s_n=0$.  Then $\lim_{n\to\infty}(s_nt_n)=0$.
  \begin{proof}
      It must be shown that for all $\epsilon>0$ there exists an $n\in\mathbb{R}$ such that $\lvert s_nt_n\rvert<\epsilon$. $\lvert s_nt_n\rvert=\lvert s_n\rvert\cdot\lvert t_n\rvert\leq\lvert s_n\rvert M$, so it is sufficient to show that $\lvert s_n\rvert M<\epsilon$ for sufficiently large $n$.  Since $\lim_{n\to\infty}s_n=0$ then for all $\frac{\epsilon}{\lvert M\rvert}$ there exists an $N\in\mathbb{R}$ such that $\lvert s_n\rvert<\frac{\epsilon}{\lvert M\rvert}$, so $\lvert s_n\rvert M\leq\lvert s_n\rvert\cdot\lvert M\rvert<\epsilon$.  So $\lvert t_n\rvert\leq M$ for all $n$ and $s_n\to0$ implies that $\lim_{n\to\infty}(s_nt_n)=0$.
  \end{proof}
 \end{Problem}
 
       \begin{Problem}
  50 3 sequences $a_n\leq s_n\leq b_n$ for all n and $\lim_{n\to\infty}a_n=\lim_{n\to\infty}b_n=s$, then $\lim_{n\to\infty}s_n=s$.
  \begin{proof}
      We know that for all $\epsilon>0$ there exists an $N_a\in\mathbb{N}$ such that $\lvert a_n-s\rvert<\epsilon$ and there exists an $N_b\in\mathbb{N}$ such that $\lvert b_n-s\rvert<\epsilon$, that is $-\epsilon+s<a_n\leq s_n\leq b_n<\epsilon+s$ for $n>\max\{N_a,N_b\}$ then $-\epsilon<a_n-s\leq s_n-s\leq b_n-s<\epsilon$ which means that $\lvert s_n-s\rvert<\epsilon$ for $n>\max\{N_a,N_b\}$ proving $lim_{n\to\infty}s_n=s$.
  \end{proof}
 \end{Problem}
 
\end{document}
